\chapter{Work Progress}
\label{chapter:work_progress}

%\section{Sensor BHI260AP Overview}

BHI260AP\footnote{Product Page: \url{https://www.bosch-sensortec.com/products/smart-sensors/bhi260ap}} is a smart sensor with integrated \acf{imu} from Bosch Sensortec. According to \cite{BoschSensor}, it includes several software functionalities, a 32-bit customer programmable microcontroller, and a 6-axis \acs{imu}. It is designed for always-on sensor applications such as fitness tracking, navigation, machine learning analytics and orientation estimation.

In this project, the idea was to use this sensor as another source of data when collecting data to train a machine learning model which would be used to help data labeling.

At the moment of this overview, three approaches were attempted:

\begin{itemize}
    \item Python \acs{api} - there was an attempt to use the Python library to communicate with the sensor but although communication was established there was a lack of documentation and examples that allowed to create code that collected data;
    \item C++ \acs{api} - there was also an attempt to use the C++ library to communicate with the sensor but the instructions available resulted in a compilation error;
    \item Development Desktop 2.0 - this is a desktop application for Windows that allows communication with all Bosch Sensortec sensors and using it it was possible to collect data.
\end{itemize}

Therefore, collecting data was only possible with the Development Desktop 2.0. However, using the Python or the C++ \acp{api} would be ideal since as they work in Linux it would be possible to use integrate their usage with \acs{ros}.