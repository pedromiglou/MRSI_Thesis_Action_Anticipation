\chapter{State of the Art}
\label{chapter:state_of_the_art}

\section{Robot Sensors}

{\color{gray}
(ideas) Write about collaborative robots and its sensors
}

\section{Algorithms}

Machine Learning algorithms have been increasingly more common in the context of action anticipation in collaborative environments. These are divided in 3 groups: Supervised Learning, Unsupervised Learning and Reinforcement Learning.

\subsection{Supervised Learning}

In \cite{Maeda2016} the authors predicted the next human action using a look-up table containing different orders for assembly actions and with the nearest neighbor algorithm the actions of the human would be matched with a certain order.

In \cite{Canuto2021} the authors used a Long Short-Term Memory (LSTM) Neural Network to handle classifying the next action using the human skeleton joints of several frames over time. These joints were obtained using OpenPose on the captured images.

\subsection{Unsupervised Learning}

\subsection{Reinforcement Learning}

\section{Order of Actions Problem}

In \cite{Maeda2016} the authors used a look-up table containing different orders for assembly actions and with the nearest neighbor algorithm the actions of the human would be matched with a certain order. The limitation of this method is that all of the possible sequences need to be on the table because if they are not there then the robot will match with a different order which may be undesirable.

{\color{gray}
review \cite{Canuto2021} way of dealing with this problem
}

\section{Human-Robot Collaboration Safety}

{\color{gray}
(ideas) Solutions when dealing with the safety of the user
}