\chapter{Conclusion}
\label{chapter:conclusion}

In this document, a review of previous work and important concepts about human-robot collaboration, and, more particularly, about action anticipation was conducted.

Regarding the sensors used, RGB cameras are the most common, although there is work with others, such as RGB-D cameras or wearables, to a lesser extent.

Regarding algorithms, machine learning techniques are predominant, given that most work nowadays takes advantage of the progress made in that field. Supervised learning received a greater focus since most of the articles use it and it is the most adequate to solve the problem in this dissertation. With the continuous evolution of machine learning, it is expected that the algorithms related to the topic in this paper also evolve and, consequently, give rise to even better solutions.

Regarding safety, this is a topic common to action anticipation since it is relevant to human-robot collaboration in general. However, as it was seen, there are norms that collaborative robots have to follow and strategies that increase the worker's safety even further.

Despite the work already done on this topic, this is still a relatively new concept, with most of the work being very recent.