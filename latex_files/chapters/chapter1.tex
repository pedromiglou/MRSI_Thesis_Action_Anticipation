\chapter{Introduction}
\label{chapter:introduction}

\section{Background}

{\color{red}
(guide) The concept of Human-Robot Collaboration (HRC) involves the study of processes in which humans and robots work together to achieve a shared goal. Research on collision avoidance, human-aware planning of robot motions and control of physical contact have brought significant advances to the field of HRC. However, an essential requirement in the control of collaborative actions is the ability to anticipate the partner’s movements and intentions. The robot should not only be able to model and predict the human’s movements but, more importantly, must anticipate them. This dissertation proposal aims to tackle these questions using state-of-the-art Machine Learning (ML) techniques.
}

{\color{gray}
(ideas) Include here definitions of anticipation, distinction between anticipation and recognition,...
}

\section{Objectives}

{\color{red}
(guide) This dissertation aims at the development of an action anticipation system to enhance human-robot collaboration in industrial settings (see Fig.1) under the AUGMANITY mobilizing project. The main tasks to be carried out can be summarized by the following points:
\begin{itemize}
\item Overview of the state-of-the-art. To provide an up-to-date review of the definitions and algorithms adopted for action anticipation in collaborative environments.
\item Action anticipation in a human-robot collaborative scenario. To formally define how to anticipate an action in the context of the collaborative task under study using RGB-D images as input. ML models, such as recurrent neural networks (RNNs), are at the forefront of the algorithms to explore.
\item Robot control of joint actions. To develop anticipatory robot controllers that consider the human partner movements and intentions and use these inferences to make appropriate decisions during the execution of a sequential assembly task.
\item Metrics and performance evaluation. To provide performance metrics used to evaluate the action anticipation models and the add-value of the anticipatory controller (e.g., in terms of cycle time).
\item Writing the master dissertation and other detailed documentation.
\end{itemize}
}